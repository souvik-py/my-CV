\documentclass[10.8pt, a4paper]{extarticle}
\usepackage{lipsum}
\usepackage{booktabs}
\usepackage{url}
\usepackage{enumitem}
\usepackage{palatino}
\usepackage{tabularx}
\fontfamily{SansSerif}
\usepackage[T1]{fontenc}
\usepackage[utf8]{inputenc}
\usepackage{multicol}
\usepackage{bibentry}
\nobibliography*
\usepackage{romannum}
\usepackage{latexsym}
\usepackage[empty]{fullpage}
\usepackage{titlesec}
\usepackage{marvosym}
\usepackage[usenames,dvipsnames]{color}
\usepackage{verbatim}
\usepackage{enumitem}
\usepackage[pdftex, hidelinks]{hyperref}
\usepackage{fancyhdr}
\usepackage{fontawesome}
\setlist[itemize]{noitemsep, topsep=0.3pt}
\usepackage{graphicx}
\graphicspath{ {./images/} }
\usepackage[charter]{mathdesign} % Bitstream Charter


\pagestyle{fancy}
\fancyhf{} % clear all header and footer fields
\fancyfoot{}
\renewcommand{\headrulewidth}{0pt}
\renewcommand{\footrulewidth}{0pt}




% Adjust margins
\addtolength{\oddsidemargin}{-0.50in}
\addtolength{\evensidemargin}{-0.250in}
\addtolength{\textwidth}{1in}
\addtolength{\topmargin}{-.5in}
\addtolength{\textheight}{1.0in}
\addtolength{\parskip}{.5mm}
\urlstyle{same}

\raggedbottom
\raggedright
\setlength{\tabcolsep}{0in}
\usepackage[margin=0.3in]{geometry}
\usepackage{xcolor, soul}
\sethlcolor{lightgray}
% Sections formatting
\titleformat{\section}{
  \vspace{-10pt}\scshape\raggedright\large
}{}{0em}{}[\color{black}\titlerule \vspace{-5pt}]

%-------------------------
% Custom commands
\newcommand{\resumeItem}[2]{
  \item\small{
    \textbf{#1}{: #2 \vspace{-2pt}}
  }
}

\newcommand{\resumeItemNoBullet}[2]{
  \item[]\small{
    \hspace{-9pt}\textbf{#1}{: #2 \vspace{-6pt}}
  }
}

\newcommand{\resumeSubheading}[4]{
  \vspace{-1pt}\item[]
  \begin{tabular*}{0.98\textwidth}{l@{\extracolsep{\fill}}r}
      \hspace{-10pt}\textbf{#1} & #2 \\
      \hspace{-10pt}\textit{\small#3} & \textit{\small #4} \\
    \end{tabular*}\vspace{-5pt}
}

\newcommand{\resumeSubItem}[2]{\resumeItem{#1}{#2}\vspace{-4pt}}

\renewcommand{\labelitemii}{$\circ$}

\newcommand{\resumeSubHeadingListStart}{\begin{itemize}[leftmargin=*]}
\newcommand{\resumeSubHeadingListEnd}{\end{itemize}}
\newcommand{\resumeItemListStart}{\begin{itemize}}
\newcommand{\resumeItemListEnd}{\end{itemize}\vspace{-5pt}}

% custom commands
\newcommand{\shorterSection}[1]{\vspace{-10pt}\section{#1}}

%-------------------------------------------
%%%%%%  CV STARTS HERE  %%%%%%%%%%%%%%%%%%%%%%%%%%%%

\renewcommand{\baselinestretch}{1.0} 
\begin{document}

\fontsize{10pt}{11.3pt}\selectfont

%----------HEADING-----------------
\vspace{0pt}

\begin{table}
    \begin{minipage}{0.15\linewidth}
        \centering
        \includegraphics[height =0.8 in]{IIT_Kanpur_Logo}
    \end{minipage}
    \begin{minipage}{0.85\linewidth}
        %\centering
        \setlength{\tabcolsep}{5 pt}
        \def\arraystretch{1.65}
        \begin{tabular}{ll}
            \textbf{\Large{Souvik Mukherjee}} \\
            Computer Science \& Engineering & 
            \hspace{2cm} 231110405\\
            Indian Institute of Technology Kanpur &  {\hspace{2cm} Languages: English, Hindi, Bengali}\\
        \end{tabular}
    \end{minipage}\hfill


\vspace{4pt}

% \begin{center}
\hspace{3 cm}
\href{mailto:souvikm23@iitk.ac.in}{\faEnvelope{ souvikm23@iitk.ac.in}} \hspace{0.2cm} \faPhone{ +91-8158920720} \hspace{0.2cm}
\href{https://github.com/souvikcseiitk}{ \faGithub{ souvikcseiitk}} \hspace{0.2cm} \href{https://www.linkedin.com/in/souvikcseiitk/}{ \faLinkedinSquare{ Souvik Mukherjee}}\hspace{0.2cm}
\href{https://www.cse.iitk.ac.in/users/souvik/}{\faBriefcase{  Portfolio}}
% \end{center}
\end{table} 

%----------- Education -----------------
\setlength{\tabcolsep}{20pt}
\begin{table}
\centering
\begin{tabular}{lllll}
\toprule
\textbf{Education}    & \textbf{University/School}   & \textbf{Subject/Discipline}    & \textbf{Year}     & \textbf{CPI/\%} \\ 
\toprule
Post Graduation & IIT, Kanpur    & MS-R, CSE (Cybersecurity)    & 2023-25   & 10.0 (**)\\ 
Graduation  & VIT, Vellore & Major,ME and Minor, CSE  & 2017-21   & 9.16\\ 
Intermediate/+2     & Sri Chaitanya, Vizag (HSC)   & STEM & 2015-17       & 92.00    \\ 
Matriculation   & Sri Chaitanya, Vizag (SSC)   & STEM    & 2015          & 10.0   \\
\bottomrule \\[-0.75cm]
\end{tabular}
\end{table}
\vspace{2pt}

%----------- Research -----------------
\shorterSection{Research Experience}  
\vspace{2pt}
\begin{itemize}
\item \textbf{
Face Morphing Attack Generation and Detection (Digital Forensics) (M.S-R Thesis);  \\Guide: Prof. Nisheeth Srivastava} \\\textbf{Impact: 10L+ students/year}
\hfill\hfill(\textit{Nov'23 - Present})

\begin{itemize}  

          \item[$\circ$]\textbf{Face Morphing:} A digital image manipulation technique that seamlessly blends two facial images, creating a fake blended face of two subjects. \\[-0.6cm]
          \item[$\circ$]\textbf{Misuse:} A deceptive tool in exams, where a morphed image combines the faces of a bright and a dull student, allowing the latter to pay the former to take exams on their behalf, thus gaining admission.
          \item[$\circ$]\textbf{Types:} Majorly two types of morphs, Landmark based and Deep-Neural-Network based (GAN/style-GAN). Done on two format's of images, JPG and PNG.
          \item[$\circ$]\textbf{Progress made:} We have achieved close to 99\% accuracy for PNG's and around 60\% for JPEG's, using Error Level Analysis (ELA) and contours in image processing. We're still working on JPEG's to improve accuracy.
        
          
    \end{itemize}
\end{itemize}



%-----------PROJECT-----------------
\vspace{4pt}
\shorterSection{Projects}
\vspace{2pt}

\begin{itemize}
  \item \textbf{DOM and DFA Attack on AES} (CS666: H/W Security for IoT)(\textbf{A Grade}) Guide: Prof. Urbi Chatterjee \href{https://github.com/souvikcseiitk/CS666-Hardware-Security-for-Internet-of-Things/tree/main/Assignment%203}{\faGithub{}}  \hfill(July'23-Nov'23)
	\\[-0.6cm]

 \begin{itemize}
\item[$\circ$] \textbf{For the DOM/DPA analysis, Objective:} Recovering AES secret key bytes using Differential Power Analysis (DPA).
\item[$\circ$] \textbf{Technique Used:} Implemented Difference of Mean Attack with zero and one bin arrays and successfully retrieved asked key bytes from power traces.
\item[$\circ$] \textbf{In DFA analysis:} We conducted the \textbf{fault injection} and \textbf{formed equations} to iteratively retrieve the key of said bytes.
\end{itemize}
\end{itemize}
\vspace{2pt}

\begin{itemize}
  \item \textbf{Packet Capture Analysis} (CS628: CSS)(\textbf{A Grade}) Guide: Prof. Angshuman Karmakar \href{https://github.com/souvikcseiitk/CS628-Computer-Systems-Security/tree/main/Assignment%204}{\faGithub{}}  \hfill(July'23-Nov'23)
	\\[-0.6cm]
	\begin{itemize}
 
          \item \textbf{Objective:} Analyzed .PCAP files for SQL injection and XSS attacks using Wireshark.
          
          \item \textbf{Methodology:}
          \begin{itemize}
          \item Filtered HTTP packets to identify potential SQL injection commands like UNION SELECT.
          \item Detected session ID theft via cookies and MD5 hashed password theft.
          \end{itemize}
          
          \item \textbf{Insights:}
          \begin{itemize}
          \item Recognized vulnerabilities in MD5 hashed passwords, susceptible to rainbow table attacks.
          \item Implemented safety measures against XSS and SQL injection attacks.
          \end{itemize}
          
          \item \textbf{Skills:} Security analysis, Wireshark, vulnerability mitigation.
          
         
\end{itemize}
\end{itemize}
\vspace{2pt}

\begin{itemize}
 \item \textbf{Designing efficient NTT, PWM \& I-NTT} (CS674 PQS)  (\textbf{A Grade}) Guide: Prof. Debapriya B. Roy \href{https://github.com/souvikcseiitk/CS674-Post-Quantum-Security}{\faGithub{}} \hfill(July'23-Nov'23)
	\\[-0.6cm]
	\begin{itemize}
	    \item[$\circ$] Firstly, we’re given 2 functions, we computed the \textbf{Fourier transform} for each one of them (using the \textbf{Cooley-Tukey NTT algorithm}). Secondly, we performed \textbf{point wise multiplication} to the transformed functions.\\[-0.6cm]

\item[$\circ$]Lastly, we did \textbf{inverse NTT} on the last output. (Using the \textbf{Gentleman-Sande inverse INTT algorithm})\\[-0.6cm]

\end{itemize}
\end{itemize}
\vspace{2pt}

\begin{itemize}
  \item \textbf{Breaking Companion ArbiteR PUF (CAR-PUF) using ML} (CS771) Guide: Prof. Purushottam Kar \href{https://github.com/souvikcseiitk/Companion-Arbiter-PUF-broken-by-ML-attacks}{\faGithub{}} \hfill(Jan'24 - Apr'24)
	\\[-0.6cm]
	\begin{itemize}
 
         \item[$\circ$] A CAR-PUF employs two arbiter PUFs, along with a secret threshold value $\tau$. Given same challenge to both, the absolute difference in timings is calculated. If $|\Delta w - \Delta r|$, is less than or equal to $\tau$, the response is 0; otherwise, it's 1.
 \\[-0.6cm]
         
        \item[$\circ$] Derived a detailed mathematical derivation demonstrating how a CAR-PUF can be compromised by a single linear model. \\[-0.6cm]

         \item[$\circ$] Wrote a code to solve this problem by learning the linear model W, b using the training data. Model used was \textbf{'model = LogisticRegression(C=1.0)'}. We mapped input features from 32 dimensions to 528, to get a proper linear fit. We had also computed how various hyper-parameters affected training time and test accuracy.
 
	

	\end{itemize}
\vspace{2pt}

\item \textbf{Escaping the Caves}(CS641)(Modern Cryptology) Guide: Prof. Manindra Agrawal \href{https://github.com/souvikcseiitk/Escaping-the-Caves}{\faGithub{}} \hfill(Jan'24 - Apr'24)
    \\[-0.6cm]
	\begin{itemize}
	      \item [$\circ$] Methodically \textbf {Analyzed and Decoded} a range of cryptosystems namely, \textbf {Substitution cipher, PlayFair cipher, EAEAE,  DES}.\\[-0.6cm]
	      
	      \item [$\circ$] Utilized advanced techniques to exploit cryptosystems, methods such as\textbf { frequency analysis, differential cryptanalysis}.\\[-0.6cm]
	\end{itemize}
\end{itemize}
\vspace{2pt}

\begin{itemize}
  \item \textbf{Project GATE CSE GPT} (Winter LLM Bootcamp, Pathway x IIT-K x IIT-BHU, Non-Academic) \href{https://github.com/souvikcseiitk/gate_cse_gpt}{\faGithub{}} \hfill(Feb'24) \\
         \textbf{Impact: 1L+ students/year} \\
         
	\begin{itemize}
 
        \item[$\circ$] A chatbot-GPT powered by \textbf{OpenAI \& Pathway}. Aims in helping students with interview, PYQ, test-series, the main exam and other common doubts, related to GATE CSE exam, specifically who are facing difficulty in affording coaching, with the help of Pathway's LLM App, and a \textbf{Dropbox} at backend\\[-0.6cm]
        
        \item[$\circ$] The LLM App enables AI-powered search from multiple unstructured documents like prev. interview experiences, PYQ's, topper's notes, etc and indexes input data in real-time just after you upload files to the cloud storage.\\[-0.6cm]


 
	

	\end{itemize}
\end{itemize}

\medskip
\vspace{-2mm}


\shorterSection{Relevant Courses and Technical Skills}
\begin{itemize}
\item \textbf{Mtech Courses :} CS771 Introduction to Machine Learning, CS641 Modern Cryptology, CS628 Computer Systems Security, CS666 Hardware Security for IOT Devices, CS674 Post Quantum Security  \\[-0.6cm]
\item \textbf{Btech Courses :} CSE2003 Data Structures \& Algorithms, CSE2004 Database Management System, CSE2001 Computer Architecture, ONL1021 Essentials of Machine Learning and Organization, CSE1003 Digital Logic and Design   \\[-0.6cm]
\item \textbf{Programming/Scripting Languages:} C, C++, Python, Java, JavaScript, Verilog HDL, HTML, CSS, MySQL.  \\[-0.6cm]
\item \textbf{ML Libraries/Utilities/Tools:} Scikit-learn, Tensorflow, PyTorch, NumPy, OpenCV, Pillow, Pandas, Matplotlib, Git, \LaTeX, Google Colab, Jupyter.  \\[-0.6cm]
 \medskip
\end{itemize}

\vspace{6mm}

%-----------Position of Responsibility / Experience-----------------

\shorterSection{Positions of Responsibility/Experience}
\begin{itemize}

\item \textbf{Head TA}, for the second semester in the \textbf{ESC111/112, Fundamentals of Computing}, which included, management of examinations and duties of other TA's, apart from the basic doubt resolving. \hfill\hfill(\textit{Jan'24-May'24})  \\[0.15cm]

\item \textbf{Teaching Assistant :} Two semesters of assisting \textbf{ESC111/112, Fundamentals of Computing} students with doubt resolution, lab test management and grading assignments\hfill\hfill(\textit{Aug'23-May'24})  \\[0.15cm]


\item \textbf{Graduate Engineer Trainee (GET) \& Associate Engineer at L\&T Technology Services } \href{https://drive.google.com/file/d/1OrfxyGyHMvoRFeCne3UPIx8iQgQaKPqk/view?usp=sharing}{\faLink{}} \href{https://drive.google.com/file/d/1JMF9GNNZ7MU2pVo0j9vEk110BOJtHZgV/view?usp=sharing}{\faLink{}} \hfill\hfill(\textit{Aug'21-March'22})  \\[-0.6cm]

\vspace{4mm}
\end{itemize}
\vspace{4mm}

%------------Achievements--------------


\shorterSection{Academic Achievements and Recognition's}
\begin{itemize}

 \item Awarded with the \textcolor{blue}{\textbf{Academic Excellence Award}} for the semester '2023-24 First' for ranking among the \textbf{top 10\%} of scorers in the department.\href{https://www.iitk.ac.in/sspc/data/2nd-list-of-Academic-Excellence-Awards-2023-21-03-24.pdf}{\faLink{}}\\[-0.6cm]
 
 
  \item My project was recognized among the \textcolor{blue}{\textbf{top 3 }}open-source projects in the Winter LLM Bootcamp cohort, offered by \textbf{Pathway X P-Club IIT Kanpur x CoPS IIT BHU} \href{https://t.certifier.io/CL0/https:%2F%2Fapi.credsverse.com%2Fv1%2Fusers%2Finvite%2Fceed2b44-26ae-4085-a1c6-291760efed0a/1/0102018ede88dc5e-89a09bd7-6cea-4807-a2c9-3dff747834d9-000000/_U00joiSf7_qp3anDSJNAiO00y7Hxm7ONFmpv6zOP2g=348}{\faLink{}}\\[-0.6cm]


    \item Selected for \textcolor{blue}{\textbf{ACM India Summer Schools 2024}}, to be held at \textcolor{blue}{\textbf{IIT Bombay}}, offered by Trust Lab, IIT Bombay. \href{https://drive.google.com/file/d/1Zrw9ZEbpcHit9Pn9m18AO0EXmoZq-p33/view?usp=sharing}{\faLink{}}
    
    Only \textbf{40 students are shortlisted from all over India}, based on profile shortlisting. 
    
    Name of the school offered: Theoretical Foundations of Cryptography \hfill\hfill(\textit{To be held from June 3 to 13, 2024}\\[-0.6cm]

  \item Attended the workshop on Data and AI with Microsoft Azure held On Campus (on 10 April 2024, at L7 (LHC))\\[-0.6cm]


\end{itemize}
\medskip



%-----------Key Courses Taken & Certifications-----------------

\shorterSection{Certifications}
\begin{itemize}

\item Supervised Machine Learning: Regression and Classification, by \textbf{deeplearning.AI \& Stanford } \href{https://www.coursera.org/account/accomplishments/certificate/TECTFLUN93UC}{\faLink{}}\\[-0.6cm]

\item Advanced Learning Algorithms, by \textbf{deeplearning.AI \& Stanford } \href{https://www.coursera.org/account/accomplishments/certificate/PLNJXZ9JXVXJ}{\faLink{}}\\[-0.6cm]

\item The joy of computing using Python, by \textbf{NPTEL, IIT Madras }  \href{https://drive.google.com/file/d/1-0Q_gA_cl3bb7sPyZOA3OtV0fbmRLQvz/view?usp=sharing}{\faLink{}}\\[-0.6cm]

\item Introduction to Generative AI, by \textbf{Google Cloud Skills }\href{https://www.cloudskillsboost.google/public_profiles/76755a5d-f18b-4787-89db-bd6be0b3e066/badges/4178524}{\faLink{}}\\[-0.6cm]

\item Data Analytics Methods for Marketing by \textbf{Meta}\href{https://www.coursera.org/account/accomplishments/certificate/JPUQ5DJBDZN6}{\faLink{}}\\[-0.6cm]

\item Participated in the \textbf{ISRO- Outreach Programme:}  and completed the following courses: 
\begin{itemize}
    \item Basics of Remote Sensing Geographical Information System and Global Navigation SatelliteSystem \href{https://drive.google.com/file/d/15E-FXqYPGD-uRhFIRUu4WKobbqnBvlAh/view?usp=sharing}{\faLink{}}
    
    \item Overview of Web GIS Technology \href{https://drive.google.com/file/d/1bQc-tYUv5at-NuImsrSm_pfTs4rRgWi2/view?usp=drive_link}{\faLink{}}
    
    \item Earth Observation for Carbon Cycle Studies \href{https://drive.google.com/file/d/1SYKP6BzNcmPcVS8MiXt31NDexVP9gFDn/view?usp=sharing}{\faLink{}}
    
\end{itemize} 

\vspace{4mm}
\end{itemize}
\vspace{4mm}

%------------Extra-Curriculars--------------
\shorterSection{Extra-Curriculars}
\begin{itemize}
\item Part Of \textbf{American Society of Mechanical Engineers (ASME) college chapter, VIT Vellore }. \href{https://m.facebook.com/story.php?story_fbid=pfbid02PFELcj2eSP5jKHCXbLVLbEHxS8VHwkXKTn5VppoYcqn2MsAU6kcHg1noYHc1g59Yl&id=406549079357847&mibextid=Nif5oz}{\faLink{}} \hfill\hfill(\textit{Dec'17-March'19}) 

I have made significant contributions in roles such as coordinator, events management, and primarily as a member of the design team. My skills encompass proficiency in Photoshop and Premiere Pro.

\item \textit{\textbf{2018, Crew member}}, \textbf{Mechnovate 2018}, an event by \textbf{ASME India, VIT Vellore} \href{https://drive.google.com/file/d/1skeeRg3UISDUtRxwI7Nfen7c6zbsuysX/view?usp=sharing}{\faLink{}}

\item \textit{\textbf{2019, Coordinator}}, \textbf{International Conference on Materials, Manufacturing and Modelling (ICMMM) 2019}, organized by the School of Mechanical Engineering (SMEC), VIT in association with ASME Chapter, VIT \href{https://drive.google.com/file/d/1skeeRg3UISDUtRxwI7Nfen7c6zbsuysX/view?usp=sharing}{\faLink{}}

\item \textit{\textbf{2019, Manager}}, \textbf{E-Fest Asia Pacific 2019}, an event by \textbf{ASME, USA} \href{https://drive.google.com/file/d/1skeeRg3UISDUtRxwI7Nfen7c6zbsuysX/view?usp=sharing}{\faLink{}} \href{https://efests.asme.org/gallery/images/e-fest-asia-pacific-2019}{\faLink{}}

\vspace{4mm}
\end{itemize}
\vspace{4mm}


%------------ National Level Exams Taken --------------
\shorterSection{National Level Exams Taken}
\begin{itemize}
\item \textbf{Graduate Aptitude Test in Engineering, GATE in CSE, All India Rank: 507, Score: 690, Percentile: 99.33}
    \begin{itemize}
    \item Fundamental of computer science and engineering, Data Structure and Algorithms, Database, Computer Architecture, Computer Networks, Operating Systems, Compiler Design, Theory Of Computation, Digital Logic and Design, Discrete \& Engineering Math.  \href{https://drive.google.com/file/d/1IkVH0v_xUTgL83VHjq7N5oFpSjw99Zma/view?usp=drive_link}{\faLink{\textbf{Scorecard}}}
    \end{itemize}
\vspace{4mm}
\end{itemize}
\vspace{4mm}

\end{document}
